% Created 2013-12-28 Sat 13:26
\documentclass[11pt]{article}
\usepackage[utf8]{inputenc}
\usepackage[T1]{fontenc}
\usepackage{fixltx2e}
\usepackage{graphicx}
\usepackage{longtable}
\usepackage{float}
\usepackage{wrapfig}
\usepackage{rotating}
\usepackage[normalem]{ulem}
\usepackage{amsmath}
\usepackage{textcomp}
\usepackage{marvosym}
\usepackage{wasysym}
\usepackage{amssymb}
\usepackage{hyperref}
\tolerance=1000
\usepackage[cm]{fullpage}
\pagestyle{empty}
\thispagestyle{empty}
\DeclareUnicodeCharacter{00A0}{~}
\author{Ethan Brown}
\date{\today}
\title{Goals 2013 Item analysis}
\hypersetup{
  pdfkeywords={},
  pdfsubject={},
  pdfcreator={Emacs 24.2.1 (Org mode 8.2.1)}}
\usepackage{Sweave}
\begin{document}

\maketitle
\begin{Schunk}
\begin{Sinput}
> source("/home/fortis/goals-2013/descriptives/descriptives.R")
> library(psych)
\end{Sinput}
\end{Schunk}

\section{Basic item analysis}
\label{sec-1}

\begin{Schunk}
\begin{Sinput}
> al <- alpha(gp.rw)
> al
\end{Sinput}
\begin{Soutput}
Reliability analysis   
Call: alpha(x = gp.rw)

  raw_alpha std.alpha G6(smc) average_r mean   sd
      0.67      0.66    0.68     0.079 0.51 0.16

 Reliability if an item is dropped:
       raw_alpha std.alpha G6(smc) average_r
q01.rw      0.66      0.65    0.67     0.079
q02.rw      0.65      0.64    0.66     0.076
q03.rw      0.66      0.66    0.67     0.081
q04.rw      0.65      0.65    0.67     0.078
q05.rw      0.67      0.67    0.68     0.083
q06.rw      0.67      0.67    0.68     0.084
q07.rw      0.64      0.64    0.65     0.075
q08.rw      0.65      0.65    0.66     0.077
q09.rw      0.65      0.65    0.66     0.077
q10.rw      0.66      0.66    0.67     0.081
q11.rw      0.66      0.66    0.67     0.079
q12.rw      0.66      0.66    0.67     0.080
q13.rw      0.64      0.64    0.65     0.074
q14.rw      0.65      0.65    0.67     0.078
q15.rw      0.66      0.66    0.68     0.081
q16.rw      0.67      0.67    0.68     0.085
q17.rw      0.66      0.65    0.67     0.079
q18.rw      0.65      0.65    0.66     0.076
q19.rw      0.66      0.66    0.67     0.080
q20.rw      0.65      0.65    0.66     0.077
q21.rw      0.65      0.64    0.66     0.076
q22.rw      0.67      0.66    0.67     0.082
q23.rw      0.66      0.65    0.67     0.079

 Item statistics 
          n    r r.cor r.drop mean   sd
q01.rw 1165 0.34 0.269  0.217 0.27 0.44
q02.rw 1165 0.44 0.401  0.328 0.64 0.48
q03.rw 1164 0.29 0.208  0.167 0.17 0.38
q04.rw 1165 0.37 0.306  0.248 0.42 0.49
q05.rw 1165 0.23 0.143  0.108 0.54 0.50
q06.rw 1164 0.19 0.092  0.068 0.73 0.45
q07.rw 1163 0.48 0.453  0.372 0.68 0.47
q08.rw 1164 0.41 0.371  0.303 0.43 0.49
q09.rw 1162 0.40 0.355  0.290 0.50 0.50
q10.rw 1163 0.30 0.218  0.176 0.74 0.44
q11.rw 1160 0.33 0.263  0.215 0.76 0.43
q12.rw 1163 0.31 0.238  0.192 0.65 0.48
q13.rw 1154 0.50 0.484  0.398 0.47 0.50
q14.rw 1152 0.38 0.323  0.266 0.63 0.48
q15.rw 1164 0.28 0.196  0.157 0.59 0.49
q16.rw 1164 0.17 0.067  0.050 0.34 0.47
q17.rw 1164 0.34 0.277  0.222 0.41 0.49
q18.rw 1163 0.42 0.386  0.312 0.35 0.48
q19.rw 1162 0.32 0.249  0.197 0.34 0.47
q20.rw 1164 0.41 0.362  0.292 0.40 0.49
q21.rw 1165 0.43 0.402  0.316 0.49 0.50
q22.rw 1163 0.26 0.197  0.142 0.64 0.48
q23.rw 1161 0.34 0.281  0.228 0.45 0.50

Non missing response frequency for each item
          0    1 miss
q01.rw 0.73 0.27 0.00
q02.rw 0.36 0.64 0.00
q03.rw 0.83 0.17 0.00
q04.rw 0.58 0.42 0.00
q05.rw 0.46 0.54 0.00
q06.rw 0.27 0.73 0.00
q07.rw 0.32 0.68 0.00
q08.rw 0.57 0.43 0.00
q09.rw 0.50 0.50 0.00
q10.rw 0.26 0.74 0.00
q11.rw 0.24 0.76 0.00
q12.rw 0.35 0.65 0.00
q13.rw 0.53 0.47 0.01
q14.rw 0.37 0.63 0.01
q15.rw 0.41 0.59 0.00
q16.rw 0.66 0.34 0.00
q17.rw 0.59 0.41 0.00
q18.rw 0.65 0.35 0.00
q19.rw 0.66 0.34 0.00
q20.rw 0.60 0.40 0.00
q21.rw 0.51 0.49 0.00
q22.rw 0.36 0.64 0.00
q23.rw 0.55 0.45 0.00
\end{Soutput}
\end{Schunk}

Coeff alpha is at 0.67, seems a bit low for a test like this.  The within-subject variance is higher than would be expected if the items were all really measuring the same thing.  I believe 0.7 is the usual cutoff for studies like this?

I'm not sure how surprising this is, since I don't necessarily expect that ``statistical reasoning'' is a unitary concept.  There's very clearly several subscales and distinct concepts that we would expect to go different directions, such as confidence intervals, study design, \emph{p}-values, etc.

I don't see any cause for concern for the changes in reliability here.

\begin{Schunk}
\begin{Sinput}
> r.cor <- al$item.stats$r.cor
> names(r.cor) <- 1:23
> barplot(rev(r.cor), horiz = T, las = 2, main = "Item discrimination (r.cor)")
> abline(v=0.2, col = "red")
\end{Sinput}
\end{Schunk}
\includegraphics{items-cttDiscrim}

Both of the herbicide items, including the supposed sample size one (q06) had very poor discrimination.  There may be some confusion about what exactly this item is saying or implying. The worst discrimination was on q16, the confidence interval/prediction for a single case (which was a relatively difficult item, as well, PC = $~ 0.34$).  q09, the candies problem, had decent discimination, q23 had acceptable, and q13 had excellent discrimination.
\section{Distractor analysis}
\label{sec-2}

\begin{Schunk}
\begin{Sinput}
> source("/home/fortis/goals-2013/functions.R")
> d06 <- distractors("q06", gp)
> d09 <- distractors("q09", gp)
> d13 <- distractors("q13", gp)
> d23 <- distractors("q23", gp)
> plq06 <- ggplot(d06, aes(pct.minus, scalecount, color = col)) +
+     geom_line() +
+     theme_bw() +
+     xlab("Proportion correct (excluding this item)") +
+     ylab("Students") +
+     ggtitle("Distractor analysis") +
+     coord_cartesian(ylim = c(0, 1)) + 
+     scale_color_discrete(name = "Sample size may be why not significant")
> print(plq06)
\end{Sinput}
\end{Schunk}

\begin{Schunk}
\begin{Sinput}
> plq06 <- ggplot(d06, aes(pct.minus, scalecount, color = col)) +
+     geom_line() +
+     theme_bw() +
+     xlab("Proportion correct (excluding this item)") +
+     ylab("Students") +
+     ggtitle("Herbicide/sample size (q06)") +
+     coord_cartesian(ylim = c(0, 1)) + 
+     scale_color_discrete(name = "Sample size may be why not significant") +
+     theme(legend.position = "top")
> print(plq06)
\end{Sinput}
\end{Schunk}
\includegraphics{items-distract06}

\begin{Schunk}
\begin{Sinput}
> ## name conflict with %+%, which is why I need to use colon notation here
> 
> plq09 <- ggplot2::"%+%"(plq06, d09) + 
+     scale_color_discrete(name = "Variability difference") +
+     ggtitle("Candies problem (q09)")
> plq09
\end{Sinput}
\end{Schunk}
\includegraphics{items-distract09}

\begin{Schunk}
\begin{Sinput}
> ## name conflict with %+%, which is why I need to use colon notation here
> 
> plq13 <- ggplot2::"%+%"(plq06, d13) +
+     scale_color_discrete(name = "CI width difference") +
+     ggtitle("CIs and sample size (q13)")
> plq13
\end{Sinput}
\end{Schunk}
\includegraphics{items-distract13}

\begin{Schunk}
\begin{Sinput}
> ## name conflict with %+%, which is why I need to use colon notation here
> 
> plq23 <- ggplot2::"%+%"(plq06, d23) + 
+     scale_color_discrete(name = "p-value comparison") +
+     ggtitle("P-value and sample size (q23)")
> plq23
\end{Sinput}
\end{Schunk}
\includegraphics{items-distract23}
% Emacs 24.2.1 (Org mode 8.2.1)
\end{document}
